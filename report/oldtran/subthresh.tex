\section{Sub-Threshold Logic}
\label{sec:subthresh}



\subsection{notes}

Operates by using leakage current as operating switching current.
Limits maximum performance of circuit.

\begin{itemize}

\item{Pros:}

\begin{itemize}

\item{Increased gm}
\item{near-ideal static noise margin}
\item{exponential power saving}

\end{itemize}

\item{Cons:}

\begin{itemize}

\item{more sensitive to power supply noise}
\item{more sensitive to temperature}
\item{more sensitive to process variations}
\item{exponential increase in delay}

\end{itemize}

\end{itemize}

\subsection{Design}
Sub-threshold logic works by reducing the supply voltage to below the threshold voltage of the transistors.
This uses the leakage current from gate to source as the switching current in the device \cite{ULPSubThresh}.
As the leakage current is exponentially related to the gate voltage, reducing the voltage to below the devices threshold voltage results in the leakage current being greatly reduced.
The knock-on effect of this is an exponential increase in delay can be expected.


\subsection{Variations}
