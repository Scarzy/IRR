\section{\acf{ULL}}
\label{sec:ull}

\subsection{notes}

Connect PMOS and NMOS transistors by their sources rather than by their drains.
Biases Vgs negatively, pushing Ioff down to physical limit.


\begin{itemize}

\item Pros:

\begin{itemize}

\item Analogue and digital compatible
\item Simple to design
\item High efficiency
\item Speed relatively unaffected
\item Factor saving in $I_{off}$ of 100-10000 \cite{ULL-AandD}

\end{itemize}

\item Cons:

\begin{itemize}

\item Increases size, each transistor becomes 2 \cite{ULL-AandD}

\end{itemize}

\end{itemize}

\subsection{Design}

Of the two main forms of loss in CMOS logic, switching and leakage currents, \ac{ULL} works to reduce the leakage currents down to the physical limits of the device.

It is often forgotten in CMOS design that there is not only digital circuitry, but also analogue components as well, as these are required to interface with the physical world.
\citeauthor{ULL-AandD} help address this by producing a number of analogue components, as well as digital ones.
Among their designs are voltage followers, transistors, and diodes.

Initially a pair of N and P channel MOSFETs are selected with appropriate $V_{Th}$ values such that their $I_{D}-V_{GS}$ curves intersect at a low current.
These transistors are then connected up in such a way that the P-channel and N-channel transistors are connected by their sources.
This is contradictory to the standard arrangement of transistors, where the transistors would connect via their drains.
The effect of connecting the transistors up this way is that the


\subsection{Variations}

