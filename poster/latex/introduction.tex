Many electronic applications require different system complexities and power usage.
There are two main forms of low power designs, one suited for batteries, one suited for energy harvesting.
\begin{itemize}
	\item Battery
	\begin{itemize}
		\item High power availability
		\item Limited energy
	\end{itemize}
	\item Energy Harvesting
	\begin{itemize}
		\item Low power availability
		\item Theoretically infinite energy
	\end{itemize}
\end{itemize}

Energy harvesting systems often require processing at minimal power expense.
They often run minimalistic tasks, designed to monitor the harvesting system.
Because their source of energy is the same source as is being harvested, the supply can be very small (possibly less than $2\mu W$).
As such, the logic has to operate with minimal draw, as otherwise it will run out of charge and fail, or it will negate the energy source.

There are two main forms of loss in logic circuits, leakage current and switching current.
\\
Leakage current
\begin{itemize}
	\item In transistors current flows between drain and source
	\item Is a small current, but is constant
	\item This is required for correct operation, however current is often larger than it needs to be
\end{itemize}

Switching current
\begin{itemize}
	\item During switching all transistors can be momentarily turned on
	\item Large currents can flow
	\item Energy dumped straight from supply to ground
\end{itemize}
